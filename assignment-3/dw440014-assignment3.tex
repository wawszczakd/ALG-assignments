\documentclass[12pt]{article}
\usepackage[utf8]{inputenc}
\usepackage[T1]{fontenc}
\usepackage{indentfirst}
\usepackage{amsmath}
\usepackage{amssymb}
\usepackage{natbib}
\usepackage{graphicx}
\usepackage{float}
\usepackage[a4paper, margin = 2 cm]{geometry}
\usepackage{fancyhdr}
\usepackage{wrapfig}
\usepackage{hyperref}
\usepackage{mathtools}

\title{Algorytmika praca domowa 3}
\author{Dominik Wawszczak}
\date{2024-05-01}

\begin{document}
	\setlength{\parindent}{0 cm}
	
	Dominik Wawszczak \hfill Algorytmika
	
	numer indeksu: 440014 \hfill praca domowa 3
	
	numer grupy: 1
	
	\bigskip
	\hrule
	\bigskip
	
	\textbf{Zadanie 1}
	
	\medskip
	
	Rozpocznijmy od udowodnienia, że dany w zadaniu problem jest w klasie NP.
	Istotnie, weryfikację czy konkretna funkcja \(f\) spełnia wszystkie krotki
	z \(\mathcal{A}\) można zrealizować w czasie wielomianowym, zwyczajnie
	sprawdzając dla każdej krotki, czy spełniona jest co najmniej jedna z
	odpowiednich nierówności.
	
	\medskip
	
	Pozostaje pokazać, że problem ten jest NP-trudny. W tym celu wskażemy
	wielomianową redukcję problemu 3-CNF-SAT do problemu z treści zadania.
	Weźmy więc dowolną instancję problemu 3-CNF-SAT. Niech \(m\) będzie liczbą
	zmiennych w tej instancji. Przyjmijmy \(n = 2m + 1\). Zmienną \(x_{i}\)
	będziemy utożsamiać z liczbą \(2i - 1\), a zaprzeczenie zmiennej \(\neg
	x_{i}\) z liczbą \(2i\). Liczbę \(2m + 1\) będziemy traktować jako sztuczny
	indykator tego czy zmienna jest prawdziwa, czy fałszywa. Konkretnie, jeśli
	\(f(2i - 1) > f(2m + 1)\), to zmienna \(x_{i}\) jest prawdziwa, a jeśli
	\(f(2i - 1) < f(2m + 1)\), to zmienna ta jest fałszywa.
	
	\medskip
	
	Rozpocznijmy z pustym zbiorem krotek \(\mathcal{A}\), do którego będziemy
	dodawać kolejne krotki. Na początek dla każdej pary \((i, j) \in \{1,
	\ldots, n\}^{2}\) takiej, że \(i < j\), wrzućmy krotkę \((i, j, j, i)\)
	gwarantując tym samym, że jeśli istnieje odpowiednia funkcja \(f\), to jest
	ona iniekcją.
	
	\medskip
	
	Kolejnym krokiem będzie zagwarantowanie, że dla każdego \(i \in \{1, \ldots,
	m\}\) zachodzi dokładnie jeden z dwóch przypadków: \(f(2i) < f(2m + 1) <
	f(2i - 1)\), czyli \(x_{i} = \top\), albo \(f(2i - 1) < f(2m + 1) < f(2i)\),
	co oznacza, że \(x_{i} = \bot\). Niech \(a, b, c \in \{1, \ldots, n\}\) będą
	parami różne. Zauważmy, że wrzucając do \(\mathcal{A}\) krotkę \((c, b, b,
	a)\) sprawiamy, że nie może zachodzić \(f(a) < f(b) < f(c)\), przy czym na
	każdym z pozostałych \(5\) możliwych posortowań \(f(a)\), \(f(b)\) i
	\(f(c)\) krotka ta jest spełniona. Możemy w ten sposób ,,zabronić''
	wszystkie kolejności  \(f(2i - 1)\), \(f(2i)\) oraz \(f(2m + 1)\), z
	wyjątkiem dwóch wymienionych wyżej.
	
	\medskip
	
	Bez straty ogólności możemy założyć, że żadna klauzula nie zawiera
	jednocześnie zmiennej i jej zaprzeczenia, gdyż takie klauzule są zawsze
	spełnione. Klauzulę z instancji problemu 3-CNF-SAT będziemy rozpatrywać jako
	zbiór liczb odpowiadających zawartym w niej literałom (powtórzenia możemy
	oczywiście zignorować), zgodnie z numeracją opisaną w drugim akapicie.
	Przykładowo, klauzulę \((x_{4} \vee \neg x_{6} \vee x_{9})\) będziemy
	rozumieć jako multizbiór \(\{7, 12, 17\}\), a klauzulę \((x_{5} \vee \neg
	x_{2} \vee \neg x_{2})\) jako \(\{4, 9\}\). Dla każdego zbioru
	jednoelementowego \(\{a\}\) dodamy do \(\mathcal{A}\) krotkę \((2m + 1, a,
	2m + 1, a)\). Dla każdego zbioru dwuelementowego \(\{a, b\}\) dodamy do
	\(\mathcal{A}\) krotkę \((2m + 1, a, 2m + 1, b)\). Wreszcie dla każdego
	zbioru trójelementowego \(\{a, b, c\}\) dodamy do \(\mathcal{A}\) krotkę
	\((2m + 1, a, b \oplus 1, c)\), zakładając bez straty ogólności \(a < b <
	c\), gdzie \(\oplus\) jest operatorem bitowym xor.
	
	\medskip
	
	Pozostaje wykazać równoważność:
	\begin{align*}
		&\text{istnieje funkcja} \ f : \{1, \ldots, n\} \to \mathbb{Q} \
		\text{spełniająca wszystkie krotki z} \ \mathcal{A} \quad \iff \\
		&\text{formuła z instancji problemu 3-CNF-SAT jest spełnialna.}
	\end{align*}
	
	\medskip
	
	Zaczniemy od wykazania implikacji ,,w lewo''. Weźmy więc wartościowanie
	spełniające formułę, po czym dla każdego \(i = 1, \ldots, m\) przypiszmy
	\(f(2i - 1) = i = f(2i) + m + 1\), jeżeli \(x_{i} = \top\), albo \(f(2i - 1)
	= i - m - 1 = f(2i) - m - 1\), w przeciwnym wypadku. \(f(2m + 1)\)
	zdefiniujemy jako \(0\). Łatwo zauważyć, że funkcja \(f\) spełnia wszystkie
	krotki z akapitów \(3\) i \(4\). Dla klauzul, którym odpowiadają zbiory
	jedno lub dwuelementowe również nietrudno zauważyć, że funkcja \(f\) spełnia
	odpowiadające im krotki. Pozostają klauzule, którym odpowiadają zbiory
	trójelementowe \(\{a, b, c\}\). Przypuśćmy nie wprost, że \(0 > f(a) \
	\wedge \ f(b \oplus 1) > f(c)\). Skoro \(b < c\) i klauzula ta nie zawiera
	jednocześnie zmiennej i jej zaprzeczenia, to \(b \oplus 1 < c\). Ponieważ
	\(f(a) < 0\), to odpowiadający jej literał jest fałszem, zatem \(f(b) > 0\)
	lub \(f(c) > 0\). Jeżeli \(f(b) > 0\), mamy:
	\[ f(b \oplus 1) \ = \ \left\lfloor \frac{b + 1}{2} \right\rfloor - m - 1 \
	< \ \left\lfloor \frac{c + 1}{2} \right\rfloor - m - 1 \ \leqslant \ f(c)
	\text{,} \]
	czyli sprzeczność. W drugim przypadku natomiast zachodzi:
	\[ f(c) \ = \ \left\lfloor \frac{c + 1}{2} \right\rfloor \ > \ \left\lfloor
	\frac{b + 1}{2} \right\rfloor \ \geqslant \ f(b \oplus 1) \text{,} \]
	co również daje sprzeczność.
	
	\medskip
	
	Na koniec pozostaje wykazać implikację ,,w prawo''. Niech więc \(f\) będzie
	funkcją spełniającą każdą klauzulę z \(\mathcal{A}\). Bez straty ogólności
	możemy założyć, że \(f(2m + 1) = 0\), po prostu odejmując stałą \(f(2m +
	1)\) od każdej wartości funkcji, co nie wpłynie na żadną nierówność. Zgodnie
	z tym, co opisano w drugim akapicie, przyjmiemy \(x_{i} = \top\), jeśli
	\(f(2i - 1) > 0\), albo \(x_{i} = \bot\), gdy \(f(2i - 1) < 0\). Od razu
	widać, że formuły zawierające jeden lub dwa literały są spełnione przy takim
	wartościowaniu. Żeby formuła trójelementowa, której odpowiada zbiór \(\{a,
	b, c\}\) była niespełniona, musiałoby zachodzić \(f(a), f(b), f(c) < 0\).
	Skoro \(f(b) < 0\), to \(f(b \oplus 1) > 0\), czyli nie zachodzi żadna z
	nierówności \(f(2m + 1) < f(a)\) i \(f(b \oplus 1) < f(c)\), co oznacza
	sprzeczność, gdyż istnieje krotka w \(\mathcal{A}\), której funkcja \(f\)
	nie spełnia.
	
	\medskip
	
	Wykazana równoważność kończy rozwiązanie zadania.
	
	\newpage
	
	\textbf{Zadanie 2}
	
	\medskip
	
	Rozpocznijmy od napisania programu liniowego do problemu z treści zadania:
	\begin{align*}
		&\underset{u \in V(G)}{\forall} \ 0 \leqslant x_{u} \leqslant 1
		\text{,} \\
		&\underset{uv \in E(G)}{\forall} \ x_{u} + x_{v} \geqslant 1 \text{,} \\
		&\underset{u \in V(G)}{\forall} \ \sum\limits_{v \in N[u]} x_{v}
		\leqslant y \text{,} \\
		&\min\!: \ y \text{.}
	\end{align*}
	Weźmy rozwiązanie tego programu w liczbach rzeczywistych, które oczywiście
	możemy uzyskać w czasie wielomianowym. Wówczas zachodzi \(y \leqslant
	\text{OPT}(G)\), ponieważ biorąc rozwiązanie programu w liczbach całkowitych
	otrzymamy optymalne rozwiązanie problemu. Niech
	\[ X \ = \ \left\{ u \ : \ x_{u} \geqslant \frac{1}{2} \right\} \text{.} \]
	Wówczas \(X\) jest pokryciem wierzchołkowym, ponieważ gdyby dla pewnej
	krawędzi \(uv \in E(G)\) zachodziło \(x_{u} < \frac{1}{2}\) i \(x_{v} <
	\frac{1}{2}\), to mielibyśmy \(x_{u} + x_{v} < 1\).
	
	\medskip
	
	Zauważmy, że dla każdego \(u \in V(G)\) spełnione jest
	\begin{equation}
		|X \cap N[u]| \ \leqslant \ 2 \cdot \sum\limits_{v \in N[u]} x_{v} \
		\leqslant \ 2 \cdot y \ \leqslant \ 2 \cdot \text{OPT}(G) \text{.}
		\tag{*} \label{eq:star}
	\end{equation}
	Wynika stąd, że wskazany algorytm jest algorytmem 2-aproksymacyjnym tego
	problemu, co kończy rozwiązanie pierwszej części zadania.
	
	\medskip
	
	Żeby gęstość pokrycia wierzchołkowego znaleziona przez nasz algorytm
	wyniosła więcej niż \(2 \cdot \text{OPT}(G) - 1\), musi istnieć wierzchołek
	\(u \in V(G)\), dla którego w każdej nierówności ciągu nierówności
	\ref{eq:star} zachodzi równość. W szczególności oznacza to, że dla pewnego
	\(u \in V(G)\) w nierówności
	\[ |X \cap N[u]| \ \leqslant \ 2 \cdot \sum\limits_{v \in N[u]} x_{v} \]
	zachodzi równość.
	
	\medskip
	
	Nierówność ta bierze się z dodania stronami nierówności postaci \(1
	\leqslant 2 x_{v}\), dla \(v \in X\), oraz \(0 \leqslant 2x_{v}\), dla
	\(v \notin X\). We wszystkich tych nierównościach musi więc zachodzić
	równość, zatem
	\[ \underset{v \in N[u]}{\forall} \ x_{v} \in \left\{ 0, \frac{1}{2}
	\right\} \text{.} \]
	
	\medskip
	
	Jeżeli dla pewnego \(v \in N[u] \setminus \{u\}\) byłoby \(x_{v} = 0\), to
	mielibyśmy \(x_{u} + x_{v} \leqslant \frac{1}{2} < 1\), czyli sprzeczność,
	toteż \(x_{v} = \frac{1}{2}\), dla każdego \(v \in N[u] \setminus \{u\}\).
	
	\medskip
	
	Jeśli \(x_{u} = 0\) oraz \(u\) nie jest wierzchołkiem izolowanym, to biorąc
	dowolnego sąsiada \(v\) otrzymujemy \(x_{u} + x_{v} = \frac{1}{2} < 1\),
	czyli również sprzeczność, z czego wynika, że \(x_{u} = \frac{1}{2}\). O
	wierzchołkach izolowanych natomiast możemy założyć, że nie ma ich w naszym
	pokryciu, ponieważ i tak są one zbędne.
	
	\medskip
	
	Z powyższego wynika, że jeżeli gęstość pokrycia wierzchołkowego jest większa
	niż \(2 \cdot \text{OPT}(G) - 1\), to istnieje wierzchołek w tym pokryciu,
	którego wszyscy sąsiedzi również w nim są. Możemy więc go zwyczajnie usunąć
	z tego pokrycia i powtarzać tę procedurę tak długo, aż taki wierzchołek
	istnieje. Na koniec zostaniemy z pokryciem o gęstości co najwyżej \(2 \cdot
	\text{OPT}(G) - 1\), co kończy rozwiązanie zadanie.
\end{document}
