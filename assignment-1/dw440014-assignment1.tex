\documentclass[12pt]{article}
\usepackage[utf8]{inputenc}
\usepackage[T1]{fontenc}
\usepackage{indentfirst}
\usepackage{amsmath}
\usepackage{amssymb}
\usepackage{natbib}
\usepackage{graphicx}
\usepackage{float}
\usepackage[a4paper, margin = 2 cm]{geometry}
\usepackage{fancyhdr}
\usepackage{wrapfig}
\usepackage{hyperref}
\usepackage{mathtools}

\title{Algorytmika praca domowa 1}
\author{Dominik Wawszczak}
\date{2024-03-09}

\begin{document}
	\setlength{\parindent}{0 cm}
	
	Dominik Wawszczak \hfill Algorytmika
	
	numer indeksu: 440014 \hfill praca domowa 1
	
	numer grupy: 1
	
	\bigskip
	\hrule
	\bigskip
	
	\textbf{Zadanie 1}
	
	\medskip
	
	Stwórzmy nowy nieskierowany graf \(G'\) taki, że
	\[ V(G') \ \coloneqq \ V(G) \ \cup \ \{t\} \quad \text{oraz} \quad E(G') \
	\coloneqq \ E(G) \ \cup \ \{xt \ : \ x \in V(G)\} \text{.} \]
	Niech funkcja kosztu \(c : V(G') \to \mathbb{Z}_{\geqslant 0}\) będzie
	następująca:
	\[ c(xy) \ \coloneqq \ \begin{cases}
		g(xy) \text{,} & \text{jeżeli} \ xy \in E(G) \text{,} \\
		f(x) \text{,} & \text{gdy} \ y = t \text{,} \\
		f(y) \text{,} & \text{dla} \ x = t \text{.}
	\end{cases} \]
	
	\medskip
	
	\underline{Lemat 1} Każdy zbiór \(X \subseteq V(G)\) zawierający \(v_{0}\)
	odpowiada pewnemu przekrojowi sieci przepływowej \((G', c, v_{0}, t)\), w
	taki sposób, że \(\text{koszt}(X)\) jest równy przepustowości tego
	przekroju.
	
	\medskip
	
	\underline{Dowód lematu 1} Dla konkretnego zbioru \(X \subseteq V(G)\)
	zawierającego \(v_{0}\) weźmiemy \(S = X\) i \(T = (V(G) \setminus X)
	\cup \{t\}\). Wtedy oczywiście \(V(G') = S \cup T\) oraz \(S \cap T =
	\emptyset\). Przepustowością tego przekroju jest wówczas
	\[ \sum\limits_{x \in S \ \wedge \ y \in T} c(xy) \ = \ \sum\limits_{x \in
	X} c(xt) + \sum\limits_{x \in X \ \wedge \ y \in V(G) \setminus X} c(xy) \ =
	\ \sum\limits_{x \in X} f(x) + \sum\limits_{x \in X \ \wedge \ y \in V(G)
	\setminus x} g(xy) \text{,} \]
	czyli \(\text{koszt}(X)\), co kończy dowód lematu.
	
	\medskip
	
	Dla dowolnego przekroju \((S, T)\) sieci \((G', c, v_{0}, t)\), biorąc \(X =
	S\) znowu otrzymamy, że \(\text{koszt}(X)\) jest równy przepustowości tego
	przekroju, zatem również każdemu przekrojowi odpowiada podzbiór wierzchołków
	grafu \(G\) zawierający \(v_{0}\).
	
	\medskip
	
	Z powyższego wynika, że minimalny możliwy koszt zbioru \(X \subseteq V(G)\)
	zawierającego \(v_{0}\) jest równy przepustowości minimalnego przekroju
	sieci \((G', c, v_{0}, t)\), czyli wartości maksymalnego przepływu tej
	sieci. Maksymalny przepływ możemy znaleźć algorytmem Dinica mającym
	złożoność czasową \(O \left( |V(G')|^{2} |E(G')| \right)\).
	
	\medskip
	
	Żeby odzyskać zbiór \(X\) wystarczy podzielić zbiór wierzchołków grafu
	\(G'\) na dwa zbiory: \(S\) -- wierzchołki do których istnieje ścieżka
	z \(v_{0}\) w sieci rezydualnej z maksymalnym przepływem oraz \(T\) --
	pozostałe wierzchołki. Można to zrealizować zwykłym algorytmem DFS w czasie
	\(O \left( |V(G')| + |E(G')| \right)\). Wówczas \(S\) jest szukanym zbiorem.
	
	\medskip
	
	Ponieważ zachodzi \(|V(G')| = |V(G)| + 1\) oraz \(|E(G')| = |E(G)| +
	|V(G)|\), to całkowita złożoność naszego algorytmu to \(O \left( |V(G)|^{2}
	(|V(G)| + |E(G)|) \right)\).
	
	\newpage
	
	\textbf{Zadanie 2}
	
	\medskip
	
	\underline{Lemat 1} Istnieje \(k\) parami rozłącznych krawędziowo ścieżek
	prostych z \(s\) do \(t\).
	
	\medskip
	
	\underline{Dowód lematu 1} Rozpatrzmy sieć przepływową \((G, \lambda xy . 1,
	s, t)\). Zauważmy, że każdemu cięciu \(Z \subseteq E(G)\) odpowiada pewien
	przekrój \((S, T)\) tej sieci, ponieważ możemy wziąć
	\[ S \ = \ \{ v \ : \ \text{istnieje ścieżka z} \ s \ \text{do} \ v \
	\text{niezawierająca krawędzi z} \ Z\} \quad \text{oraz} \quad T \ = \ V(G)
	\setminus S \text{.} \]
	Jeśli dodatkowo cięcie \(Z\) jest minimalne, czyli \(|Z| = k\), to każda
	krawędź \(xy \in Z\) spełnia warunek \(x \in S \ \wedge \ y \in T\), gdyż
	w przeciwnym razie mamy jeden z dwóch przypadków:
	\begin{itemize}
		\item \(x \in T\), wówczas krawędź \(xy\) jest redundantna i istnieje
		mniejsze cięcie \(Z \setminus \{xy\}\), ponieważ i tak nie da się dojść
		z \(s\) do \(x\);
		\item \(y \in S\), wtedy podobnie krawędź \(xy\) nic nie zmienia i
		istnieje mniejsze cięcie \(Z \setminus \{xy\}\), albowiem krawędź ta nie
		nakłada żadnych ograniczeń, jako że i tak istnieje inna ścieżka z \(s\)
		do \(y\).
	\end{itemize}
	W analogiczny sposób każdemu przekrojowi odpowiada pewne cięcie, z czego
	wynika, że przepustowość minimalnego przekroju wynosi \(k\), skąd wartość
	maksymalnego przepływu również wynosi \(k\). Na wykładzie było dowodzone, że
	przepływ ten rozkłada się na ścieżki i cykle, przy czym cykle możemy po
	prostu zignorować, a o ścieżkach założyć, że na każdej z nich wartość
	przepływu jest całkowita, czyli wynosi \(1\), ponieważ algorytm
	Forda-Fulkersona jest poprawny. Wnioskujemy stąd, że ścieżki te są parami
	rozłączne krawędziowo, gdyż w przeciwnym wypadku istniałaby krawędź, dla
	której funkcja przepływu przekraczałaby funkcję przepustowości. Otrzymaliśmy
	rodzinę \(k\) parami rozłącznych krawędziowo ścieżek prostych z \(s\) do
	\(t\): \(P = \{(s = v_{1, 1}, v_{1, 2}, \ldots, v_{1, l_{1}} = t), (s =
	v_{2, 1}, v_{2, 2}, \ldots, v_{2, l_{2}} = t), \ldots, (s = v_{k, 1}, v_{k,
	2}, \ldots, v_{k, l_{k}} = t)\}\), co kończy dowód lematu.
	
	\medskip
	
	\underline{Lemat 2} Każde \(k\)-cięcie zawiera po dokładnie jednej krawędzi
	na każdej ścieżce z \(P\) i nie zawiera żadnych innych krawędzi.
	
	\medskip
	
	\underline{Dowód lematu 2} Każde \(k\)-cięcie musi zawierać po co najmniej
	jednej krawędzi na każdej ścieżce z \(P\), ponieważ w przeciwnym wypadku nie
	byłoby cięciem. Ponadto, jeśli zawiera dwie krawędzie z którejś ścieżki lub
	pewną krawędź nienależącą do żadnej ze ścieżek, to ma rozmiar większy niż
	\(k\), czyli sprzeczność, co kończy dowód lematu.
	
	\medskip
	
	Niech \(A \subseteq V(G)\) będzie zbiorem zawierającym takie wierzchołki
	\(v\), że w dowolnym \(k\)-cięciu zachodzi \(v \in S\), gdzie \((S, T)\)
	jest odpowiadającym temu cięciu przekrojem. Podobnie definiujemy \(B\), z
	tym że zbiór ten zawiera wierzchołki, które dla każdego \(k\)-cięcia są w
	\(T\).
	
	\medskip
	
	\underline{Lemat 3} Dla każdego \(i = 1, 2, \ldots, k\) istnieje \(p_{i} \in
	\{ 1, 2, \ldots, l_{i} - 1\}\) takie, że \(v_{i, 1}, v_{i, 2}, \ldots, v_{i,
	p_{i}} \in A\) oraz \(v_{i, p_{i} + 1}, v_{i, p_{i} + 2}, \ldots, v_{i,
	l_{i}} \notin A\).
	
	\medskip
	
	\underline{Dowód lematu 3} Nie ulega wątpliwości, że \(s = v_{i, 1} \in A\).
	Przypuśćmy nie wprost, że istnieją \(a, b \in \{1, 2, \ldots, l_{i} - 1\}\)
	takie, że \(a < b\) oraz \(v_{i, a} \notin A \ \wedge \ v_{i, b} \in A\).
	Weźmy \(k\)-cięcie \(Z\), dla którego \(v_{i, a} \in T\). Z lematu \(2\)
	wnioskujemy, że dla pewnego \(c \in \{1, 2, \ldots a - 1\}\) zachodzi
	\(v_{i, c} v_{i, c + 1} \in Z\), przy czym jest to jedyna krawędź na tej
	ścieżce należąca do \(Z\). Oczywiście \(v_{i, b} \in S\), zatem istnieje
	pewna ścieżka z \(s\) do \(v_{i, b}\) niezawierająca krawędzi z \(Z\).
	Możemy tę ścieżkę przedłużyć o \((v_{i, b}, v_{i, b + 1}, \ldots, v_{i,
	l_{i}} = t)\) uzyskując ścieżkę z \(s\) do \(t\) niezawierającą krawędzi z
	\(Z\), czyli sprzeczność, co kończy dowód lematu.
	
	\medskip
	
	\underline{Lemat 4} Dla każdego \(i = 1, 2, \ldots, k\) istnieje \(q_{i} \in
	\{ 2, 3 \ldots, l_{i}\}\) takie, że \(v_{i, q_{i}}, v_{i, q_{i} + 1},
	\ldots, v_{i, l_{i}} \in B\) oraz \(v_{i, 1}, v_{i, 2}, \ldots, v_{i, q_{i}
	- 1} \notin B\).
	
	\medskip
	
	\underline{Dowód lematu 4} Analogiczny jak lematu 3.
	
	\medskip
	
	Niech
	\[ Z_{A} \ = \ \{v_{i, p_{i}} v_{i, p_{i} + 1} \ : \ i = 1, 2, \ldots, k\}
	\quad \text{oraz} \quad Z_{B} \ = \ \{v_{i, q_{i - 1}} v_{i, q_{i}} \ : \ i
	= 1, 2, \ldots, k\} \text{.} \]
	
	\medskip
	
	\underline{Lemat 5} \(Z_{A}\) i \(Z_{B}\) są \(k\)-cięciami.
	
	\medskip
	
	\underline{Dowód lematu 5} Przypuśćmy nie wprost, że istnieje ścieżka \((s =
	v_{1}, v_{2}, \ldots, v_{l} = t)\), taka że żadna z krawędzi \(v_{j} v_{j +
	1}\) nie należy do \(Z_{A}\), dla \(j = 1, \ldots, l - 1\). Weźmy największe
	takie \(p \in \{1, 2, \ldots, l - 1\}\), że \(v_{1}, v_{2}, \ldots, v_{p}
	\in A\). Wtedy z lematu 3 krawędź \(v_{p} v_{p + 1}\) nie leży na żadnej
	ścieżce z \(P\), zatem z lematu 2 nie jest w żadnym \(k\)-cięciu, skąd
	wnioskujemy, że \(v_{p + 1} \in A\), ponieważ \(v_{p} \in A\), czyli
	sprzeczność, toteż \(Z_{A}\) jest \(k\)-cięciem. Analogicznie dowodzimy, że
	\(Z_{B}\) również jest \(k\)-cięciem, co kończy dowód lematu.
	
	\medskip
	
	Z założeń \(Z_{A} \cap Z_{B} \neq \emptyset\), co oznacza, że istnieje
	\(i\), dla którego \(p_{i} + 1 = q_{i}\). Krawędź \(v_{i, p_{i}} v_{i,
	q_{i}}\) jest więc szukaną krawędzią, ponieważ gdyby istniało \(k\)-cięcie
	\(Z\) takie, że \(v_{i, p_{i}} v_{i, q_{i}} \notin Z\), to spełnione by było
	\(v_{i, q_{i}} \in S\), gdyż \(v_{i, p_{i}} \in A\), co przeczy warunkowi
	\(v_{i, q_{i}} \in B\). Otrzymana sprzeczność kończy rozwiązanie zadania.
\end{document}
